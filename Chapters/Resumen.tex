\chapter*{Resumen}
\label{chapter:Resumen}
\markboth{RESUMEN}{}
\index{Resumen}
\addcontentsline{toc}{chapter}{Resumen}

Las herramientas de acoplamiento molecular han llegado a ser bastante eficientes en el descubrimiento de fármacos y en el desarrollo de la investigación de la industria farmacéutica. Estas herramientas se utilizan para elucidar la interacción de una pequeña molécula (ligando) y una macromolécula (diana) a un nivel atómico para determinar cómo el ligando interactúa con el sitio de unión de la proteína diana y las implicaciones que estas interacciones tienen en un proceso bioquímico dado. El progreso experimentado en las técnicas de acoplamiento molecular ha estado a la par con los avances en los métodos espectroscópicos biomoleculares como la cristalografía de rayos X y la resonancia magnética nuclear (NMR), que han sido muy importantes en el dominio de la biología estructural. Estas técnicas han permitido determinar más de 100.000 estructuras tridimensionales de proteínas que pueden tener un papel importante en las rutas de bioseñalización. La base de datos de \emph{Protein Data Bank} actualmente contiene 130.807 estructuras \emph{PDB} de múltiples organismos, la mayoría de ellos habiendo sido obtenidos a través de cristalografía de rayos X (117.083), NMR (11.766) y cristalografía de electrón (1.545). En este contexto, en que el existen miles de estructuras PDB almacenadas que pueden ser candidatas a ser analizadas como dianas terapéuticas, las técnicas de acoplamiento molecular juegan un papel importante en el diseño de nuevos fármacos analizando cómo estos interactúan con las dianas terapéuticas a nivel molecular.

En el desarrollo computacional de las herramientas de acoplamiento molecular los investigadores de este área se han centrado en mejorar los componentes que determinan la calidad del \emph{software} de acoplamiento molecular: 1) la función objetivo y 2) los algoritmos de optimización. La función objetivo de energía se encarga de proporcionar una evaluación de las conformaciones entre el ligando y la proteína calculando la energía de unión, que se mide en kcal/mol. Según la literatura, existen varios tipos de funciones objetivo de energía pero la mayoría de ellos están basados en campos de fuerza que estiman la energía libre de unión de la conformación ligando-receptor, teniendo en cuenta términos como las conformaciones del ligando interno, las conformaciones proteína-ligando y los efectos solventes. En esta memoria, hemos usado AutoDock, ya que es una de las herramientas de acoplamiento molecular más citada y usada, y cuyos resultados son muy precisos en términos de energía y valor de RMSD (desviación de la media cuadrática). Además, se ha seleccionado la función de energía de AutoDock versión 4.2, ya que permite realizar una mayor cantidad de simulaciones realistas incluyendo flexibilidad en el ligando y en las cadenas laterales de los aminoácidos del receptor que están en el sitio de unión.

En esta tesis se han utilizado algoritmos de optimización para mejorar los resultados de acoplamiento molecular de AutoDock 4.2, el cual minimiza la energía libre de unión final que es la suma de todos los términos de energía de la función objetivo de energía. Dado que encontrar la solución óptima en el acoplamiento molecular es un problema de gran complejidad y la mayoría de las veces imposible, se suelen utilizar algoritmos no exactos como las metaheurísticas, para así obtener soluciones lo suficientemente buenas en un tiempo razonable.

Por todo lo anterior, como trabajo preliminar se puede analizar el rendimiento de un conjunto de metaheurísticas mono-objetivo de carácter general (en su diseño canónico) para determinar si es posible obtener mejores valores de la función objetivo que con aquellas técnicas proporcionadas por AutoDock. Según la literatura consultada, existen pocos estudios que tengan en cuenta la flexibilidad en sus experimentos de acoplamiento molecular. Es por ello, que se aplicó flexibilidad tanto en los ligandos como en las cadenas laterales de las macromoléculas. De esta manera, es posible determinar el rendimiento de los algoritmos atendiendo si el espacio de búsqueda es diferente o no dependiendo del tamaño del ligando y su flexibilidad.

Dados los interesantes resultados obtenidos por Janson \emph{et al.} (2008) en el que se minimizaron dos objetivos, la energía intermolecular ($E_{inter}$) y la intramolecular ($E_{intra}$), se puede ver que el problema puede ser formulado usando dos objetivos contrapuestos, dando lugar a un problema de optimización multi-objetivo. Después de revisar el resto de la literatura sobre los distintos enfoques multi-objetivo para resolver el acoplamiento, se observó que todos los estudios estaban basados en la función de energía de AutoDock 3.0 (una versión anterior a AutoDock 4.2), que no aplica flexibilidad a las cadenas laterales de los aminoacidos del receptor y, por lo tanto, solamente se hicieron simulaciones siendo rígidas la macromolécula y el ligando o con flexibilidad sólo en el ligando. Estos estudios también habían sido realizados sobre un conjunto pequeño de problemas, con lo que estudios con un mayor número de complejos flexibles podrían dar lugar a resultados muy interesantes.

Los estudios multi-objetivo anteriormente propuestos no han considerado anteriormente guiar la búsqueda usando uno de los objetivos cuando la estructura del ligando co-cristalizado es conocida, lo que podría completar la función de energía tradicional. Podrían planificarse nuevos enfoques utilizando este hecho como punto de partida. También se hipotetiza que este enfoque puede ser útil en aquellos estudios \emph{in silico} que tengan que ver con la selección de nuevos compuestos anticancerígenos para dianas terapeutas que sean resistentes a múltiples fármacos.

El objetivo principal de esta tesis es explorar un enfoque al problema del acoplamiento molecular que pueda dar lugar a un conjunto más amplio de soluciones dependiendo de los objetivos seleccionados. Con esto, se intenta promover el uso de estas nuevas técnicas en lugar de depender en los algoritmos más comúnmente usados. Como trabajo previo, se aplican nuevas técnicas mono-objetivo que puedan proporcionar resultados de mayor calidad que las técnicas usualmente aplicadas.

Las fases que se siguieron en el desarrollo de esta tesis fueron las siguientes:

\begin{enumerate}
	
	\item Exploración del estado del arte actual sobre los estudios de acoplamiento molecular e investigación de las diferentes herramientas usadas y análisis del código de AutoDock 4.2, dado que es la más citada y popular entre la comunidad científica. Se observaron las técnicas de optimización que proporcionaba AutoDock y se estudió la posibilidad de añadir nuevos algoritmos que mejoraran los resultados obtenidos.
	
	\item Para conseguir este objetivo, en lugar de intentar incorporar los nuevos algoritmos directamente en el código fuente de AutoDock, se utilizó un \emph{framework} orientado a la resolución de problemas de optimización con metaheurísticas. Concretamente, se usó jMetal, que es una librería de código libre basada en Java. Ya que AutoDock está implementado en C++, se desarrolló una versión en C++ de jMetal. De esta manera, se consiguió integrar ambas herramientas (AutoDock 4.2 y jMetal) para optimizar la energía libre de unión entre compuesto químico y receptor.
	
	\item Después de disponer de una amplia colección de metaheurísticas implementadas en jMetalCpp, se realizó un detallado estudio en el cual se aplicaron un conjunto de metaheurísticas para optimizar un único objetivo minimizando la energía libre de unión, el cual es el resultado de la suma de todos los términos de energía de la función objetivo de energía de AutoDock 4.2. Por lo tanto, cuatro metaheurísticas tales como dos variantes de algoritmo genético gGA (Algoritmo Genético generacional) y ssGA (Algoritmo Genético de estado estacionario), DE (Evolución Diferencial) y PSO (Optimización de Enjambres de Partículas) fueron aplicadas para resolver el problema del acoplamiento molecular. Esta fase se dividió en dos subfases en las que usamos dos conjuntos de instancias diferentes, utilizando como receptores HIV-proteasas con cadenas laterales de aminoacidos flexibles y como ligandos inhibidores HIV-proteasas flexibles. El primer conjunto de instancias se usó para un estudio de configuración de parámetros de los algoritmos y el segundo para comparar la precisión de las conformaciones ligando-receptor obtenidas por AutoDock y AutoDock+jMetalCpp.
	
	\item La siguiente fase implicó aplicar una formulación multi-objetivo para resolver problemas de acoplamiento molecular dados los resultados interesantes obtenidos por Janson \emph{et al.} (2008) en que dos objetivos como la energía intermolecular ($E_{inter}$) y la energía intramolecular ($E_{intra}$) fueron minimizados. Por lo tanto, se comparó y analizó el rendimiento de un conjunto de metaheurísticas multi-objetivo mediante la resolución de complejos flexibles de acoplamiento molecular minimizando la $E_{inter}$ y la $E_{intra}$. Estos algoritmos fueron: NSGA-II (Algoritmo Genético de Ordenación No dominada) y su versión de estado estacionario (ssNSGA-II), SMPSO (Optimización Multi-objetivo de Enjambres de Partículas con Modulación de Velocidad), GDE3 (Tercera versión de la Evolución Diferencial Generalizada), MOEA/D (Algoritmo Evolutivo Multi-Objetivo basado en la Decomposición) y SMS-EMOA (Optimización Multi-objetivo Evolutiva con Métrica S). Estos algoritmos han obtenido rendimientos satisfactorios en una amplia variedad de problemas de optimización, sin embargo, nunca se han usado con anterioridad para resolver problemas de acoplamiento molecular a excepción del algoritmo NSGA-II.
	
	\item Después de probar enfoques multi-objetivo ya existentes, se probó uno nuevo. En concreto, el uso del RMSD como un objetivo para encontrar soluciones similares a la de la solución de referencia. Se replicó el estudio previo usando este conjunto diferente de objetivos.
	
	\item Por último, se analizó de forma detallada el algoritmo que obtuvo mejores resultados en los estudios previos. En concreto, se realizó un estudio de variantes del SMPSO minimizando la $E_{inter}$ y el RMSD. SMPSO aplica un mecanismo de limitación de la velocidad de las partículas para impedir el movimiento de éstas en las regiones de búsqueda ajenas a los rangos de los problemas. Este algoritmo usa un archivo externo para almacenar las soluciones no dominadas según a su distancia de \emph{crowding}. También se usa este archivo en el mecanismo de selección del líder. Este estudio proporcionó algunas pistas sobre cómo nuevos algoritmos basados en SMPSO pueden ser adaptados para mejorar los resultados de acoplamiento molecular para aquellas simulaciones que involucren ligandos y receptores flexibles.
	
\end{enumerate}

Resumiendo, esta tesis realiza las siguientes contribuciones:

\begin{itemize}
	
	\item La implementación de un framework metaheurístico en C++ (jMetalCpp), versión del ampliamente usado framework en Java jMetal, para resolver problemas de optimización y para su posterior distribución pública entre la comunidad científica.
	
	\item La inclusión de técnicas metaheurísticas de jMetalCpp en la herramienta de acoplamineto molecular AutoDock, y su distribución pública para incrementar las posibilidades a los usuarios de ámbito biológico cuando resuelvan el problema del acoplamiento molecular.
	
	\item La demostración de que el uso de técnicas de optimización mono-objetivo diferentes aparte de aquéllas ampliamente usadas en las comunidades de acoplamiento molecuolar podría dar lugar a soluciones de mayor calidad. En nuestro caso de estudio, el algoritmo de evolución diferencial obtuvo mejores resultados que aquellos obtenidos por AutoDock.
	
	\item La propuesta de diferentes enfoques multi-objetivo para resolver el problema del acoplamiento molecular, tales como la decomposición de los términos de la energía de unión o el uso del RMSD como un objetivo.
	
	\item La demostración del SMPSO, una metaheurística de optimización multi-objetivo de enjambres de partículas, como una técnica remarcable para resolver problemas de acoplamiento molecular cuando se usa un enfoque multi-objetivo, obteniendo incluso mejores soluciones que las técnicas mono-objetivo.
	
	\item La presentación de dos nuevas variantes de SMPSO. La primera es SMPSOD, una aproximación sin archivo, que está inspirada en el MOEA/D. La segunda es SMPSOC, que usa la nueva similaridad del coseno para calcular el estimador de densidad.
	
\end{itemize}

El problema del acoplamiento molecular es una de las técnicas usadas en el proceso de diseño de fármacos basados en estructura. Este proceso consiste en estudios \emph{in silico} para determinar compuestos químicos que puedan ser posibles candidatos para dianas terapeúticas. Son muchas las técnicas computacionales que se utilizan adicionalmente al acomplamiento molecular, algunas de éstas son dinámica molecular y screening virtual basado en estructuras. Aparte del proceso de diseño de fármacos basados en estructura, existe otro basado en el diseño de estructuras basado en ligandos que consiste en testear librerías de compuestos químicos activos para la detección de posibles dianas terapéuticas. 

Como anteriormente se ha mencionado, el principal objetivo del problema de acoplamiento molecular es encontrar la conformación ligando-receptor cuya energía de unión sea mínima. Esta energía se computa utilizando la función de energía del \emph{software} de acoplamiento molecular. La solución que representa la interacción ligando-receptor está codificada por un vector de números reales de tamaño $n$+7 en el cual los tres primeros valores corresponden a los valores de los tres ejes ($x$, $y$, $z$) en el espacio de coordenadas Cartesianas, los siguientes cuatro valores corresponden a la orientación ligando/macromolécula, y los $n$ valores restantes son los ángulos dihedrales de torsión para el ligando y las cadenas laterales de los aminoácidos del receptor. En los experimentos realizados para esta tesis doctoral, se aplicó una metodología basada en el tamaño de malla implementada en AutoDock versión 4.2. La malla corresponde al espacio de búsqueda en el que se realiza los cómputos ligando-macromolécula en las simulaciones de acoplamiento molecular. Los parámetros utilizados fueron para ($x$, $y$, $z$) 120 y 0,375\AA\ de espacio de malla. Estos parámetros para la malla fueron suficientes para abarcar toda la superficie molecular de la macromolécula. Sin embargo, estos parámetros pueden ser modificados por el experto en acoplamiento molecular aumentando o disminuyendo tales parámetros en el espacio de malla.

Para el enfoque de optimización mono-objectivo, se minimizó el valor de la energía libre de unión, que se mide en kcal/mol. Cuanto más pequeño es este valor, más estable es el complejo ligando-receptor en términos energéticos. Atendiendo a la función de energía proporcionada por AutoDock, este valor es el resultado de la suma de la diferencia los estados de unión y no unión del ligando, receptor y del complejo ligando-receptor. Cada par de términos de evaluación incluyen evaluaciones de dispersión/repulsión, enlaces de van der Waals, puentes de hidrógeno, fuerzas de torsión e interacciones electrostáticas y de solvatación.

Para el enfoque de optimización multi-objetivo, en primer lugar, se optimizaron dos energías: la $E_{inter}$ y $E_{intra}$. La primera energía representa la diferencia entre los estados de unión y desunión del ligando-receptor o el estado energético del complejo ligando-receptor. La segunda energía representa los estados de unión y desunión del ligando y el receptor, respectivamente. Esta energía involucra la deformidad desde el punto de vista de energía de los elementos de interacción durante las simulaciones de acoplamiento molecular. Esta estrategia de optimización multi-objectivo es muy útil en aquellos casos en los que el experto tiene que elegir una solución en el conjunto de soluciones obtenidas en la que el ligando sea más más estable en términos de energía o bien, otra solución en la que el complejo ligando-receptor es más estable energéticamente.

En una segunda estrategia, se optimizó la $E_{inter}$ y el valor de RMSD calculado a partir del ligando co-cristalizado y el computado. Este valor mide la calidad de los resultados obtenidos en las simulaciones del acoplamiento molecular. RMSD básicamente es una medida de la distancia media entre las coordenadas atómicas ($x$, $y$, $z$) de la estructura del ligando co-cristalizado y el ligando computado. Esta medida tiene en cuanta la simetría, la simetría parcial (por ejemplo, la simetría de una parte rotable de la molécula) y la simetría más próxima. La comunidad científica usa el límite de 2\AA\ para distinguir entre resultados más o menos exactos. Esta medida es muy útil en aquellos casos en los que la estructura de ligando es conocida, es decir, la estructura cristalográfica del ligando con respecto al receptor está disponible en las bases de datos que almacenan estructuras cristalográficas (como la base de datos PDB). Es importante mencionar, que una estructura computada con un valor RMSD de 0\AA\ no es la mejor solución que se podría obtener ya que el receptor puede tener otros sitios de unión no conocidos y estos podrían ser interesantes desde un punto de vista farmacológico.


\section*{Trabajos publicados}

El trabajo realizado en esta tesis ha dado lugar a varias publicaciones y divulgaciones científicas. Específicamente, cuatro artículos han sido publicados en revistas indexadas en el \emph{Journal of Citation Report} (JCR) del \emph{Institute of Scientific Information}. Además, otros cuatro artículos han sido publicados en congresos. Dos de ellos se publicaron en congresos internacionales y los otros dos en congresos nacionales. Para ver más detalle, véase el Capítulo~\ref{chapter:publishedWorks}.

A continuación se resumen los artículos que avalan esta tesis. Todos estos artículos están relacionados con la aplicación de optimizaciones tanto mono-objetivo como multi-objetivo para resolver el problema del acoplamiento molecular. En el primer artículo se describió la integración de AutoDock y jMetal y su aplicación en el acoplamiento molecular. En el segundo artículo publicado, se realiza un estudio comparando las técnicas mono-objetivo usando un conjunto de instancias flexibles. En el tercer estudio, se aplica un conjunto de metaheurísticos multi-objetivo para optimizar dos objetivos, guiando al algoritmo en su búsqueda de las mejores soluciones.

\subsection*{jMetalCpp: optimizing molecular docking problems with a C++ metaheuristic framework}

En este artículo se presentó jMetalCpp, la version C++ de jMetal, el framework de metaheurísticas originalmente programado en Java. También se presenta la combinación de este software con el ampliamente usado AutoDock. Como se ha mencionado anteriormente, ambos paquetes software fueron publicados en la web para ser libremente usados por la comunidad científica.

%En~\cite{lopez14jmetalcpp} se presenta jMetalCpp, la version C++ de jMetal, el framework de metaheurísticas originalmente programado en Java. También se presenta la combinación de este software con el ampliamente usado AutoDock. La inclusión de jMetalCpp en AutoDock proporcionó a éste último varias técnicas metaheurísticas adicionales para resolver problemas de acoplamiento molecular. Ambos paquetes software (el jMetalCpp autónomo y la integración de jMetalCpp con AutoDock) fueron publicados en la web\footnote{http://jmetalcpp.sourceforge.net/}\footnote{http://khaos.uma.es/autodockjmetal/} para ser libremente usados por la comunidad científica.


\subsection*{Solving molecular flexible docking problems with metaheuristics: a comparative study}

%En nuestro primer estudio mono-objetivo~\cite{lopez2015asoc}, se testeó el rendimiento de nuevas técnicas metaheurísticas aparte de aquellas incluidas en las herramientas de AutoDock para resolver problemas de acoplamiento molecular. AutoDock proporciona dos técnicas diferentes para resolver el problema: un Algoritmo Genético Lamarckiano (LGA), que incluye búsqueda local, y un algoritmo génetico común (GA). Se añadieron cuatro metaheurísticas mono-objetivo: gGA (Algoritmo Genético generacional) y ssGA (Algoritmo Genético de estado estacionario), DE (Evolución Diferencial) y PSO (Optimización de Enjambres de Partículas). Se realizó un estudio usando 75 complejos proteína-ligando obtenidos de PDB, haciendo uso de las misma función objetivo y los mismos parámetros de configuración de AutoDock para realizar una comparación lo más justa posible. El objetivo fue la energía de unión (en kcal/mol) asociadas con el complejo ligando-receptor, como se explica en la sección~\ref{subsection:mono_solution}. Por lo tanto, cuanto más negativo es valor de la energía de unión, mejor es el resultado.

En este trabajo, se demostró que DE (jMetal) obtuvo los mejores resultados en 67 de las 75 instancias estudiadas, seguido por LGA (AutoDock que consiguió los mejores resultados en las ocho instancias restantes (1B6L, 1BDL, 1HEF, 1HIV, 1HPO, 1K6C, 1Z1H and 1ZIR). Estos resultados fueron proporcionados con confianza estadística ($\alpha = 0.05$) ya que se aplicó una serie de tests estadísticos no paramétricos. En concreto, se calcularon los \emph{ranking de Friedman} y los tests multicomparativos de Holm, y mostraron que el DE consiguió un mejor rendimiento estadísticamente que el resto de los algoritmos analizados. Este hecho es remarcable que los algoritmos de AutoDock están específicamente diseñados para resolver problemas de acoplamiento molecular. También se observó que el DE mostraba un comportamiento de convergencia más lento, aunque tendiendo a soluciones más exitosas que sus competidores. Sin embargo, gGA demostró tener una rápida convergencia, y también consiguió soluciones de alta calidad, así que este algoritmo podría ser una buena opción cuando se buscara una alternativa que proporcionara soluciones lo suficientemente buenas en un tiempo de cómputo menor.

\subsection*{A new multi-objective approach for molecular docking based on RMSD and binding energy}

Este trabajo fue presentado en la 3ª \emph{International Conference on Algorithms for Computational Biology} (AlCoB 2016), que se celebró en Trujillo (España) en junio de 2016. Dicho trabajo derivó de la idea de aplicar un enfoque de optimización multi-objetivo para resolver problemas de acoplamiento molecular. Al principio, la estrategia que se siguió fue la descomponer la energía final de unión (el objetivo a minimizar en el trabajo anterior) en varias componentes, concretamente las energías intra e intermolecular. Posteriormente, se decidió usar como objetivos la misma energía tomada como objetivo en el estudio mono-objetivo y el RMSD. Estos conceptos están explicados en más detalle en la Sección~\ref{subsection:multi_solution}.

%Sin embargo, en esta publicación~\cite{LopezCamacho2016AlCoB}, se seleccionaron cuatro algoritmos de optimización multi-objetivo representativos: NSGA-II, GDE3, SMPSO y MOEA/D. Se usó un \emph{benchmark} compuesto de 11 complejos incluyendo flexibilidad tanto en el ligando como en el receptor. Concretamente, se aplicaron 10 enlaces rotables a los ligandos y 6 enlaces rotables a la ARG-8 de las cadenas laterales de las macromoléculas (HIV-proteasas). La selección de estos complejos estuvo motivada por el amplio rango de tamaños de ligando (inhibidores de tamaño pequeño, medio, grande e inhibidores del ciclo de la urea) que caracterizaba dicho \emph{benchmark}. Se calcularon dos indicadores de calidad para medir el rendimiento de cada algoritmo: el Hipervolumen (I$_HV$) y el indicador Epsilon Aditivo Unario ($I_{\epsilon+}$). El primer indicador tiene en cuenta tanto la convergencia como la diversidad de las soluciones, mientras que el segundo sólo proporciona una medida del grado de convergencia de las aproximaciones del frente de Pareto obtenido. Cabe destacar que, se está tratando con problemas de optimización del mundo real, los frentes de Pareto verdaderos que se necesitan para calcular estas métricas no son conocidos (dado que se conocen las soluciones óptimas), así que tienen que ser obtenido usando todos los frentes aproximados de todas las ejecuciones de todos los algoritmos multi-objetivo para cada problema.

El $I_{HV}$ es la suma del volumen contribuido de cada punto de un frente con respecto a un punto de referencia, así que cuanto más alto el grado de convergencia y diversidad de un frente, más alto será el valor del hipervolumen. Según estos resultados, SMPSO consiguió los mejores valores de $I_{HV}$ en los 11 problemas, siendo MOEA/D la segunda técnica que obtuvo mejores resultados. Es importante destacar que muchos algoritmos obtuvieron un valor de $I_{HV}$ igual a cero. Esto ocurre cuando todos los puntos de los frentes producidos están situados más allá de los límites del punto de referencia. Este hecho se da en la mayoría de los problemas en todos los algoritmos a excepción de SMPSO, lo que lleva a pensar que se está afrontando un problema de optimización de gran complejidad. SMPSO también consigue el mejor rendimiento según el indicador $I_{\epsilon+}$ (en este caso, cuanto más bajo es el valor, mejor es). SMPSO alcanza los mejores valores para todas las  instancias exceptuando el 1HTF donde consiguió el segundo mejor valor. MOEA/D (que fue el que obtuvo el mejor resultado para la instancia 1HTF) alcanzando los segundos mejores valores para 9 instancias. GDE3 consiguió el segundo mejor valor en la instancia restante (1HPX), mientras que NSGA-II obtuvo los peores resultados para todas las instancias.

Después de que se presentara este trabajo, se invitó a ser substancialmente extendido y enviado al número especial de la revista \emph{IEEE/ACM Transactions on Computational Biology and Bioinformatics} (TCBB, Factor de impacto JCR 2014: 1.438, Cuartil Q1). Hasta el día de hoy, aún sigue en revisión.

\subsection*{A study of archiving strategies in multi-objective PSO for molecular docking}

%Este trabajo~\cite{GarciaNieto2016ANTS} fue presentado en el décimo \emph{International Conference on Swarm Intelligence} (ANTS 2016), celebrado en Bruselas (Bélgica) en septiembre de 2016. Este trabajo de investigación es la continuación natural del trabajo previo, donde se obtuvo que SMPSO alcanzó los mejores resultados aplicando un enfoque multi-objetivo para resolver problemas de acoplamiento molecular. El experimento anterior fue replicado usando varias variantes de SMPSO basadas en diferentes estrategias de archivo. Las variantes escogidas fueron: SMPSO$_{hv}$, SMPSOD y SMPSOC. El SMPSO original y el OMOPSO (el algoritmo del que se inspiró SMPSO) también fueron incluidos en la comparación.

Este artículo presentó la variante denominada SMPSOC, que se caracteriza por el uso de la similaridad por coseno cuando se calcula el valor de densidad de cada punto en el frente de soluciones. La variante SMPSOD también fue presentada en este artículo por primera vez. Es un enfoque sin archivo, que está implementado como una versión agregativa de SMPSO inspirado por MOEA/D.

Según el indicador $I_{HV}$, SMPSO$_{hv}$ obtuvo los mejores resultados para las 11 instancias, mientras que SMPSOD tuvo los segundos mejores en 6 instancias, SMPSOC en tres y el SMPSO original en dos, respectivamente. De igual forma, SMPSO$_{hv}$ obtuvo de nuevo los mejores resultados en las 11 instancias según el indicador $I_{\epsilon+}$. Los segundos mejores valores fueron conseguidos por SMPSOD en 7 instancias, por el SMPSO original en tres y por SMPSOC en una instancia, respectivamente.

\section*{Conclusiones y trabajos futuros}

Al abordar problemas de acoplamiento molecular, las técnicas disponibles para resolverlos no han cambiado en los últimos años. Como estos problemas pueden ser formulados como problemas de optimización multiobjetivos, nuestra intención fue la de estudiar y proporcionar un conjunto de técnicas metaheurísticas modernas para resolverlas. Como la herramienta de acoplamiento molecular más utilizada (AutoDock) fue programada en C++, nos embarcamos en la tarea de crear una versión del \emph{framework} metaheurístico jMetal en este lenguaje: jMetalCpp. De esta manera, hemos proporcionado a la comunidad de investigación una herramienta potente y de código abierto que se puede utilizar libremente.

La implementación del \emph{framework} jMetalCpp proporciona ventajas a los investigadores, tanto en el descubrimiento de fármacos como en otros dominios de las ciencias de la vida, que están interesados en disponer de técnicas más modernas que les ayudarán a resolver diferentes problemas como el acoplamiento molecular. Ya hemos demostrado que existen diferentes técnicas aparte de las que se utilizan comúnmente para resolver problemas de acoplamiento molecular y que pueden conducir a resultados de mayor calidad. La inclusión de jMetalCpp en la ampliamente utilizada herramienta AutoDock proporciona a otros investigadores una colección de metaheurísticas y herramientas adicionales a las que ya están incluidas en Autodock. También proporciona una estructura fácil para usuarios más avanzados con habilidades de programación en C++ para incorporar sus propias técnicas para resolver problemas de acoplamiento molecular. Esta herramienta está disponible \emph{online} y ya ha sido descargada por investigadores de diferentes partes del mundo. El \emph{framework} jMetalCpp independiente también está disponible para los investigadores que quieran utilizarla para resolver problemas de optimización de otros dominios. Se ha descargado cientos de veces de todo el mundo y hemos estado en contacto con personas que querían contribuir al código añadiendo sus propias herramientas y algoritmos, y utilizarlo en sus propios trabajos de investigación.

Usando AutoDock+jMetal, se realizó un estudio utilizando metaheurísticas mono-objetivo donde incluimos más algoritmos (aparte de los ya incluidos por AutoDock) para resolver un gran \emph{benchmark} de complejos proteína-ligando. El estudio se llevó a cabo teniendo los mismos parámetros de configuración que comúnmente se utilizaron en las publicaciones de AutoDock. Probamos que otras metaheurísticas mono-objetivo podrían llevar a resultados de mayor calidad. En nuestro caso, el algoritmo de evolución diferencial demostró ser un mejor candidato a la hora de resolver problemas de acoplamiento molecular.

Cuando se abordan problemas de acoplamiento molecular, es común resolverlos adoptando un enfoque mono-objetivo. Sin embargo, cuando se utiliza un enfoque multi-objetivo, un conjunto de soluciones se devuelve al final de una ejecución en lugar de una única solución. Este conjunto de soluciones ofrece al usuario final varias posibilidades desde donde escoger dependiendo del peso que quiere dar a cada uno de los objetivos de optimización. Por lo tanto, hemos considerado dos enfoques multi-objetivos diferentes en nuestros estudios. La primera se basó en la descomposición de la energía de unión final (la función objetivo que es minimizada por los algoritmos mono-objetivo) en varios componentes. Se seleccionaron las energías intra e intermoleculares como objetivos de optimización. Esto resultó en un conjunto de soluciones en las que el usuario final podría elegir dependiendo de la importancia que le da a cada una de las energías.

La otra formulación multi-objetivo utilizó el mismo objetivo que la formulación mono-objetivo (la energía de unión) y el RMSD. El uso del RMSD como objetivo para guiar la búsqueda es útil en aquellos casos típicos en los que el sitio activo de una diana terapéutica dada muta y lo hace resistente a múltiples fármacos. Utilizando este enfoque, se devuelve un amplio conjunto de soluciones, que pueden seleccionarse de acuerdo con el peso de la RMSD y la energía de unión, en lugar de centrarse únicamente en los valores de energía. Se realizó un primer estudio utilizando cuatro algoritmos multi-objetivo: NSGA-II, SMPSO, GDE3 y MOEA/D. En este experimento, se seleccionó un conjunto de 11 complejos de proteína-ligando heterogéneos con ligandos y receptores flexibles como instancias del problema. SMPSO proporcionó el mejor rendimiento general según los dos indicadores de calidad utilizados ($I_{HV}$ y $I_{\epsilon+}$) y para las instancias moleculares estudiadas, siendo MOEA/D el algoritmo con los segundos mejores valores. Así mismo, desde un punto de vista mono-objetivo, las soluciones obtenidas de SMPSO fueron mejores que las obtenidas por el algoritmo LGA de AutoDock. Esto es bastante notable ya que SMPSO es un algoritmo de optimización de propósito general, mientras que LGA está específicamente adaptado para hacer frente al problema de acoplamiento molecular. Finalmente, es interesante notar que SMPSO convergió a la región del frente que minimiza más el objetivo RMSD, mientras que MOEA/D colocó sus soluciones en la región opuesta de los frentes generados de soluciones no dominadas.

A partir de los resultados obtenidos en el último estudio, se llevó a cabo un nuevo experimento en el que se probarían varias variantes SMPSO con diferentes estrategias de archivo. Las variantes seleccionadas fueron: SMPSO$_{hv}$, SMPSOD y SMPSOC. El SMPSO original y OMOPSO (el algoritmo del que SMPSO se inspiró) también se incluyeron en la comparación. El estudio multi-objetivo anterior se replicó utilizando estos seis algoritmos y las mismas configuraciones que antes. De acuerdo con nuestros dos indicadores habituales de calidad ($I_{HV}$ y $I_{\epsilon+}$), SMPSO$_{hv}$ demostró obtener los mejores valores, seguido de SMPSOD, SMPSOC y SMPSO. La primera variante obtuvo el mejor $I_{HV}$ al realizar un método de selección de líder de aquellas soluciones no dominadas (del archivo externo) con las mayores contribuciones de hipervolumen, las cuales parecían ser responsables de los mejores valores de diversidad y convergencia en esta comparación. OMOPSO mostró resultados moderados, aunque alcanzando superar las soluciones atípicas para algunos casos. Cabe destacar que la variante SMPSOD fue capaz de cubrir el frente de referencia con soluciones no dominadas en los extremos de los dos objetivos (valores bajos de energía y bajos valores de RMSD, respectivamente).

La línea de estudio llevada a cabo en esta tesis nos ha llevado a planificar varios trabajos posibles. Por un lado, algunas de los trabajos futuros surgen de la idea de continuar el problema abordado (acoplamiento molecular) y todavía se centran en tratar de mejorar la calidad de los resultados obtenidos. Por otro lado, las nuevas líneas de investigación podrían partir de los conocimientos obtenidos en los experimentos anteriores y podrían considerarse como ``ramas'' de este trabajo.

El primer trabajo planeado está relacionado con nuestro primer estudio multi-objetivo, el cual obtuvo que al unir las soluciones generadas a partir de los algoritmos SMPSO y MOEA/D se cubría todo el frente de Pareto. Como trabajo futuro, esto nos llevó a pensar que una implementación híbrida de SMPSO y MOEA/D nos proporcionaría un conjunto más amplio de soluciones que cubriría el frente de referencia con soluciones no dominadas en los dos extremos de los objetivos. Los resultados obtenidos por SMPSOD en el segundo estudio multi-objetivo nos animaron a continuar este plan de trabajo.

En relación con el diseño del algoritmo híbrido, planeamos implementar e incluir en jMetalCpp algunos operadores específicamente diseñados para el problema de acoplamiento molecular. Hasta ahora, todas las técnicas metaheurísticas que hemos utilizado en nuestros estudios utilizan operadores de variación de propósito general, por lo que es natural llegar a la conclusión de que si las técnicas utilizadas para resolver el acoplamiento molecular están específicamente diseñadas para este problema concreto, podríamos obtener una mayor calidad de soluciones.

Otra contribución a la comunidad científica que queremos explorar es la creación de un servicio Web que proporcione las mismas herramientas que jMetalCpp integra en AutoDock. Este servicio Web permitiría ejecuciones de acoplamiento molecular utilizando todas las metaheurísticas de jMetalCpp en un complejo proteína-ligando (seleccionable de todos nuestros conjuntos anteriores o cargado por el usuario). Esta idea surgió ya que algunos usuarios con un perfil más biológico podrían tener problemas tratando de compilar y ejecutar nuestra herramienta AutoDock+jMetal.

Finalmente, como una idea más general, querríamos usar nuestro \emph{framework} jMetalCpp independiente para resolver otros problemas en las ciencias de la vida, y no estar restringidos a acoplamiento molecular. Nuestra herramienta es lo suficientemente abstracta para incluir más algoritmos y ser utilizada para resolver otros problemas de optimización de diferentes dominios. En concreto, la predicción de estructura terciaria de proteínas es un candidato muy adecuado donde aplicar el conjunto de técnicas de optimización de jMetalCpp.