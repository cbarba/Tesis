\chapter{Conclusions and Future Work}
\label{chapter:conclusions}

This chapter exposes the final ideas of this dissertation. Section~\ref{sec:conclusions} contains the conclusions obtained in all the past experiments. Then, in Section~\ref{sec:futureWorks} we explained the future lines of work that we plan to explore from the latter works.

\section{Conclusions}
\label{sec:conclusions}

When tackling molecular docking problems, the available techniques to solve them have not changed over the last years. As these problems can be formulated as multi-objective optimization problems, our intention was to study and provide a set of modern metaheuristic techniques to solve them. As the most used molecular docking tool (AutoDock) was coded in C++, we embarked on the task of the creation of a port of the metaheuristic framework jMetal in this language: jMetalCpp. This way, we have provided the research community with a powerful and open-source tool that can be freely used.

The implementation of the jMetalCpp framework provides advantages to researchers both in drug discovery and other life sciences domains who are interested in having more modern techniques that will help them to solve different problems like molecular docking. We have already demonstrated that different techniques exist apart than the ones that are commonly used to solve molecular docking problems and that they can lead to higher quality results. The inclusion of jMetalCpp into the widely used tool AutoDock provides other researchers with a collection of metaheuristics and tools additional from those that are already included in Autodock. It also provides an easy structure for more advanced users with C++ coding skills to incorporate their own techniques to solve molecular docking problems. This tool is publicly online and has been already downloaded by researchers from different parts of the world. The standalone jMetalCpp framework is also available for researchers to be used for solving optimization problems of other domains. It has been downloaded hundreds of times from all the world\footnote{1,917 downloads from SourceForge at the present day} and we have been in contact with people who wanted to contribute to the code adding their own tools and algorithms, and use it in their own research work.

Using AutoDock+jMetal, a study was done using single-objective metaheuristics where we included more algorithms (apart from those already included by AutoDock) to solve a large benchmark of protein-ligand complexes. The study was carried on taking the same configuration parameters that commonly were used in the AutoDock publications. We proved that other single-objective metaheuristics could lead to higher quality results. In our case, the differential evolution algorithm proved to be a better candidate when solving molecular docking problems.

When tackling molecular docking problems using a multi-objective approach, a set of solutions  is returned at the end of one execution instead of a single solution. This set of solutions provides the end user with several possibilities from where to choose depending of the weight she/he wants to give to each of the optimization objectives. So, we have considered two different multi-objective approaches in our studies. The first one was based on decomposing the final binding energy (the objective function that is minimized by the single-objective algorithms) into several components. We selected the intra- and inter-molecular energies as optimization objectives. This resulted in a set of solutions where the end user could select from depending on the importance that he gives to each one of the energies.

The other multi-objective formulation used the same objective as the single-objective formulation (the binding energy) and the RMSD. The use of RMSD as objective to guide the search is useful in those typical cases in which the active site of a given therapeutic target mutates and makes it multi-drug resistant. Using this approach, a broad set of solutions are returned, which can be selected according to the weight of the RMSD and binding energy, instead of only focusing on energy values. A first study was made using four multi-objective algorithms: NSGA-II, SMPSO, GDE3 and MOEA/D. In this experiment, an heterogeneous set of 11 protein-ligand complexes with flexible ligands and receptors were selected as problem instances. SMPSO provided the best overall performance according to the two quality indicators used ($I_{HV}$ and $I_{\epsilon+}$) and for the studied molecular instances, being MOEA/D the algorithm with second best values. Also, from a single-objective point of view, the solutions obtained from SMPSO were better than those obtained from the LGA algorithm form AutoDock. This was remarkable as SMPSO is a general purpose optimization algorithm, whereas LGA is specifically adapted to deal with the molecular docking problem. Finally, it is interesting to note that SMPSO converged to the region biased towards the RMSD objective, whereas MOEA/D placed its solutions in the opposite region of the generated fronts of non-dominated solutions.

From the results obtained in the last study, a new one was carried on where several SMPSO variants with different archiving strategies would be tested. The selected variants were: SMPSO$_{hv}$, SMPSOD and  SMPSOC. The original SMPSO and OMOPSO (the algorithm which SMPSO was inspired from) were also included in the comparison. The previous multi-objective study was replicated using these six algorithms and the same configurations than before. According to our two usual quality indicators ($I_{HV}$ and $I_{\epsilon+}$), SMPSO$_{hv}$ was revealed to obtain the best values, followed by SMPSOD, SMPSOC and SMPSO. The former variant obtained the best $I_{HV}$ as it included a leader selection method of those non-dominated solutions (from the external archive) having the largest hypervolume contributions, which seemed to be responsible of the best diversity and convergence values in this comparison. OMOPSO showed moderated results, although reaching outperforming outlier solutions for some instances. It is worth noting that SMPSOD variant was able to cover the reference front with non-dominated solutions in the two objectives extremes (low energy and low RMSD values respectively).

We can summarize the research work of this thesis in the fact that the use of modern multi-objective algorithms can provide the biologists with accurate solutions for the molecular docking problems. The use of these more modern variants of SMPSO instead of the common used tecniques for solving the molecular docking problem have been demonstrated to achieve better results.

\section{Future Work}
\label{sec:futureWorks}

The line of study carried on in this dissertation has lead us to plan several possible works. On one hand, some of future works emerge from the idea of continuing the tackled problem (molecular docking) and still focus on trying to improve the quality of the obtained results. On the other hand, new research lines could be started from the knowledge obtained in the previous experiments and could be considered as ``branches'' of this work.

The first planned work is related to our first multi-objective study, which obtained that joining the solutions generated from SMPSO and MOEA/D algorithms covered the full Pareto front. As a future work, this leaded us to think that a hybrid implementation of SMPSO and MOEA/D would provide us with a broader set of solutions that would cover the reference front with non-dominated solutions in the two objective ends. The results obtained by SMPSOD in the second multi-objective study encouraged us in continuing this plan of work.

Related to the hybrid algorithm design, we plan to implement and include into jMetalCpp some operators that are specifically designed to the molecular docking problem. Until now, all the metaheuristic techniques that we have used in our studies use general purpose variation operators (crossover and mutation), so it is natural to get the conclusion that if the techniques used to solve the molecular docking are specifically designed to this concrete problem we could obtain higher quality solutions.

Other contribution to the scientific community that we want to explore is the creation of a Web service that provides the same tools that jMetalCpp integrates to the AutoDock tools. This Web service would allow molecular docking executions using all the jMetalCpp metaheuristics on one protein-ligand complex (selectable from all our previous sets or uploaded by the user). This idea emerged as some users with a more biological background could have problems trying to compile and execute our AutoDock+jMetal tool.

We also plan to work in the automatic design of algorithms in order to develop ad-hoc metaheuristics that could lead to better solutions according to the optimization objectives. Some preliminary work has already been tackled on the automatic design of algorithms but for general purpose problems. It is in our interest to apply these advances for solving the molecular docking problem.

Finally, as a more general idea, we would want to use our standalone jMetalCpp framework to solve other problems in the life sciences, and not be restricted to only molecular docking. Our tool is  abstract enough to include more algorithms and to be used to solve other optimization problems from different domains. In particular, the tertiary protein structure prediction is a very promising candidate to apply the set of jMetalCpp optimization techniques.

%\section{Main contributions}
%\label{sec:conclusions-contributions}
%
%To summarize, the main contributions of this dissertation are as follows: \hl{IGUALES QUE EN LA INTRODUCCION, CAMBIAR, RESUMIR O QUITAR.}
%
%\begin{itemize}
%	\item The implementation of a C++ metaheuristic framework (jMetalCpp), port of the widely used Java framework jMetal, to solve optimization problems and its later public distribution between the scientific community.
%	\item The inclusion of the metaheuristic techniques from jMetalCpp into the molecular docking tool AutoDock, and its public distribution for increasing the possibilities of biological user when tackling the molecular docking problem.
%	\item The demonstration that different single-objective optimization techniques apart from those widely used between the molecular docking communities could lead to a higher quality results. In our case of study, DE obtained better results than those obtained by AutoDock.
%	\item The proposal of different multi-objective approaches to solve the molecular docking problem, such as the binding energy decomposition or the use of RMSD as an objective.
%	\item The demonstration of SMPSO as a good technique to solve molecular docking problems when taking a multi-objective approach and even achieving better solutions than the usual single-objective techniques.
%
%\end{itemize}
